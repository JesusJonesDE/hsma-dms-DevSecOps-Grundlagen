\section{Historische Entwicklung von DevOps zu DevSecOps}
Dieses Kapitel zeigt die Entstehung der DevOps-Kultur auf und wie der logische Schritt zu DevSecOps vollzogen wurde.

\subsection{Entstehung von DevOps und der Weg zu DevSecOps}
Der Begriff DevOps entstand Ende der 2000er Jahre und beschreibt die engere Zusammenarbeit von Software-Entwicklung und IT-Betrieb. In der Vergangenheit – und teilweise auch heute noch – führten die Kluft zwischen Abteilungen und unklare Zuständigkeiten oft zu Problemen, die erst dann sichtbar wurden, wenn eine Abteilung, die nicht in die Entwicklung involviert war, eine Software oder ein Softwaresystem mit meist manuellen Schritten in den Produktivbetrieb übernehmen sollte. Durch das DevOps-Prinzip können solche Probleme frühzeitig erkannt werden, insbesondere durch ein T-Shaped-Entwicklungsteam, in dem auch Kenntnisse im Betrieb von Software vorhanden sind.

DevOps ist als eine Kultur und Praxis zu verstehen, die Entwicklung und Betrieb integriert. Ziel ist es, durch automatisierte Prozesse und verbesserte Zusammenarbeit eine qualitativ hochwertige und reproduzierbar betreibbare Software schneller bereitzustellen.

\subsection{TBD: Zusammenarbeit zwischen Entwicklung und Betrieb}
Traditionell arbeiteten Entwicklung und Betrieb isoliert, was zu Kommunikationsproblemen und Verzögerungen führte. DevOps fördert die enge Zusammenarbeit beider Teams, um schnell auf Marktveränderungen zu reagieren und die Softwarequalität zu verbessern.


\subsection{TBD: Prinzipien der kontinuierlichen Integration und kontinuierlichen Bereitstellung (CI/CD)}
Kontinuierliche Integration (CI): Entwickler integrieren regelmäßig ihren Code in ein zentrales Repository, gefolgt von automatisierten Builds und Tests, um Fehler frühzeitig zu erkennen und zu beheben.

Kontinuierliche Bereitstellung (CD): Automatisiert die Bereitstellung von Softwareänderungen in verschiedene Umgebungen, sodass jede erfolgreiche Änderung schnell und zuverlässig an Endnutzer ausgeliefert werden kann.
