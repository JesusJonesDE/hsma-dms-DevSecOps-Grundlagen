\documentclass[conference,compsoc,final,a4paper]{IEEEtran}
\usepackage[utf8]{inputenx}
\linespread{1.25}


%% Bitte legen Sie hier den Titel und den Autor der Arbeit fest
\newcommand{\autoren}[0]{Kari, Jonas \newline Rittershofer, Daniel}

\newcommand{\dokumententitel}[0]{DevSecOps - Implementierung von Quality-Gates zur Förderung der Sicherheit von Software-Artefakten während der Entwicklung}

\input{preambel} % Weitere Einstellungen aus einer anderen Datei lesen

\begin{document}

% Titel des Dokuments
\title{\dokumententitel}

% Namen der Autoren
\author{
  \IEEEauthorblockN{\autoren}
  \IEEEauthorblockA{
    Hochschule Mannheim\\
    Fakultät für Informatik\\
    Paul-Wittsack-Str. 10,
    68163 Mannheim
    }
}

% Titel erzeugen
\maketitle
\thispagestyle{plain}
\pagestyle{plain}

% Eigentliches Dokument beginnt hier
% ----------------------------------------------------------------------------------------------------------

% Kurze Zusammenfassung des Dokuments
\begin{abstract}
TBD

\end{abstract}

% Inhaltsverzeichnis erzeugen
{\small\tableofcontents}

% Abschnitte mit \section, Unterabschnitte mit \subsection und
% Unterunterabschnitte mit \subsubsection
% -------------------------------------------------------
 Softwareentwicklung zu gewährleisten.


% -------------------------------------------------------
\section{Einleitung in DevSecOps}
In der heutigen Zeit gewinnt das DevOps-Prinzip im Softwareentwicklungszyklus zunehmend an Bedeutung. DevOps fördert die Integration von Entwicklung und Betrieb und verringert die bisherige Trennung zwischen diesen Bereichen ~\cite{Sharma2013}.  Dieser Ansatz wird durch das Prinzip "'you build it, you run it"' verkörpert, welches die Verantwortlichkeit für den gesamten Lebenszyklus einer Anwendung in die Hände der Entwickler legt.

\begin{figure}[h!]
  \includegraphics[width=\linewidth]{img/DevSecOps.png}
  \caption{DevSecOps Zyklus ~\cite{Haug2020}}
  \label{fig:devsecops}
\end{figure}

In traditionellen Softwareunternehmen, ist Sicherheit meist ein separater Schritt, welcher Mehraufwand und somit Kosten verursacht. In modernen DevOps orientierten Unternehmen ist dies nicht der Fall, dort wird Sicherheit als Geschäftsfaktor gesehen und somit eher als Feature mit Mehrwert anstelle einer Last~\cite{Wong2024}.

Dies ist sowohl im Hinblick auf die Vermeidung von ungeplanter Arbeit und Nacharbeit als auch während des eigentlichen Verkaufsprozesses, wenn Sicherheitsanforderungen als Teil einer Sicherheitsbewertung des Anbieters festgelegt werden, von Bedeutung.

Dieser Ansatz, Sicherheitsaspekte frühzeitig und kontinuierlich in den Softwareentwicklungsprozess zu integrieren, wird als "'shift left"' bezeichnet. Dies bedeutet, dass Sicherheitsmaßnahmen nicht erst am Ende des Produktzyklus, sondern bereits während der Entwicklung berücksichtigt werden müssen ~\cite{Haug2020}.

\begin{figure}[h!]
  \includegraphics[width=\linewidth]{img/ShiftLeft.png}
  \caption{Shift Left der Sicherheit ~\cite{Haug2020}}
  \label{fig:shiftleft}
\end{figure}

In dieser Arbeit liegt der Schwerpunkt auf der Implementierung von Quality Gates, die darauf abzielen, die Sicherheit von Software zu erhöhen. Insbesondere wird der Fokus auf das Dependency Scanning zur Identifikation bekannter Sicherheitslücken und die statische Code-Analyse gelegt, um potenzielle Fehler und Schwachstellen im Code frühzeitig zu erkennen.

Durch diese Maßnahmen wird nicht nur die Sicherheit der Software verbessert, sondern auch die Effizienz des gesamten Entwicklungsprozesses gesteigert. Die Bedeutung dieser Ansätze wird im Kontext moderner Softwareentwicklung erläutert und durch praxisnahe Beispiele und Fallstudien untermauert.


Diese Arbeit richtet sich an Software-Entwickler und IT-Fachleute, die DevOps und DevSecOps einführen möchten. Ziel ist es, die Sicherheit und Qualität ihrer Software-Releases zu verbessern. Leser sollten grundlegendes Wissen über Softwareentwicklung und IT-Betrieb haben und daran interessiert sein, Sicherheits- und Qualitätsstrategien sowie Security-Gates in ihren Entwicklungsprozess zu integrieren. Die Arbeit bietet anhand von zwei Beispielen eine Implementierung sowie Best Practices, um eine sichere und effiziente  Softwareentwicklung zu gewährleisten.
\section{Security by Design}
Security by Design kann als eines der Kernprinzipien von DevSecOps betrachtet werden. Es besteht aus den in den folgenden Unterkapiteln beschriebenen Bestandteilen.

\subsection{Frühe Integration}
Die frühe Integration von Sicherheitsaspekten stellt sicher, dass diese bereits in der initialen Design- und Architekturplanungsphase berücksichtigt werden. Dazu werden Sicherheitsspezialisten in diese Phasen involviert, um potenzielle Sicherheitsrisiken und -lücken zu identifizieren und sichere Systeme zu entwickeln. Wenn von Anfang an ein Fokus auf Sicherheit gelegt wird, können spätere Anpassungen vermieden werden, welche deutliche Mehrkosten verursachen würden, wie bereits unter "shift left" beschrieben.

\subsection{Threat Modeling}
Im Rahmen von sogenannten Gefahren Modellierungen,  werden potenzielle Bedrohungen und Schwachstellen identifiziert, um ein Verständnis dafür zu entwickeln, wie ein Angreifer das System kompromittieren könnte. Die Entwickler begeben sich selbst also auf die Seite eines potentiellen Angreifers und versuchen ihre eigene Anwendung anzugreifen. Diese Simulation ermöglicht es den Entwicklern ein Gefühl dafür zu entwickeln, wie potentielle Angreifer agieren könnten und ermöglicht somit die Planung von Verteidigungsmaßnahmen.

\subsection{Sichere Programmierstandards und -praktiken}
Die Implementierung und Durchsetzung sicherer Programmierstandards und -praktiken zur Vermeidung von gängigen Schwachstellen wie SQL-Injection, Cross-Site-Scripting (XSS) und Buffer Overflows ist ein wichtiger Baustein um Anwendungen sicherer zu gestalten. Dies beinhaltet regelmäßige Code-Reviews oder bspw. statische Code-Analyse welche in einem späteren Kapitel noch näher betrachtet wird.

\subsection{Continuous Monitoring}
Continuous Monitoring bezeichnet die kontinuierliche Überwachung von Anwendungen und Infrastruktur hinsichtlich sicherheitsrelevanter Schwachstellen und Vorfälle. Dies beinhaltet den Einsatz von Sicherheitsinformations- und Ereignismanagement-Systemen (SIEM), Intrusion Detection/Prevention-Systemen (IDS/IPS) sowie weiterer Monitoring-Tools. Da Angreifer einen höheren Schaden verursachen können, je länger sie unentdeckt bleiben, ist es essentiell durch Continuous Monitoring diese schnellstmöglich zu erkennen.

\subsection{DevSecOps Culture}
Die DevSecOps-Kultur zielt darauf ab, eine Kultur der gemeinsamen Verantwortung für die Sicherheit über die Grenzen von Entwicklung, Betrieb und Sicherheit hinweg zu fördern. Dies beinhaltet regelmäßige Trainings- und Sensibilisierungsprogramme für Entwickler und Betriebspersonal zu Sicherheits best Practices. Wenn sich jede Person die Teil des Softwareentwicklungsprozesses ist für Sicherheit mit Verantwortlich fühlt, können Fehler schneller gefunden und behoben werden.

\subsection{Configuration Management}
Das Configuration Management gewährleistet eine sichere Konfiguration aller Komponenten des Software-Stacks, einschließlich Server, Datenbanken und Netzwerkgeräte. Dies erfolgt durch den Einsatz von Configuration-Management-Tools zur Durchsetzung von Sicherheitsrichtlinien sowie regelmäßige Überprüfungen der Konfigurationen. Somit kann sichergestellt werden, das jeder Entwickler sich an die Vorgegebenen Sicherheitsstandards hält. Eine Fehler in der Konfiguration eines einzelnen Entwicklers kann schnell dazu führen, dass ein gesamtes System komprimiert wird, denn ein Angreifer durch einen einzelnen Eintrittspunkt sich durch das gesamte System arbeiten kann.

\subsection{Incident Response Planning}
Incident Response Planning bezeichnet die Vorbereitung auf potenzielle Sicherheitsvorfälle durch die Entwicklung und Prüfung von Notfallplänen. Dadurch kann das Team schnell und effektiv auf Sicherheitsverletzungen reagieren und diese beheben. Das selbe Prinzip ist bereits seit Langem beispielsweise bei Feuerprobealarmen oder Probeeinsätzen der Fall. Durch Übung des Ernstfalles kann Sichergestellt werden, dass alle Beteiligten routiniert sind und Chaos vermieden wird.


\subsection{Compliance and Governance}	
Die Sicherstellung der Konformität des Software-Entwicklungsprozesses mit relevanten Industriestandards, Regularien und Best-Practice-Ansätzen ist Gegenstand der Compliance- und Governance-Aktivitäten. Dies beinhaltet regelmäßige Audits und Assessments zur Überprüfung der Compliance. Im Idealfall werden diese Audits durch externe Drittanbieter vollzogen, um eine Unvoreingenommene Einschätzung zu erhalten.
\section{Integration von Sicherheitspraktiken in den DevOps-Zyklus}
In diesem Kapitel wird anhand von statischer Code-Analyse und Dependency Scanning aufgezeigt, wie Sicherheitslücken erkannt und behoben werden können, bevor diese in für den Produktionsbetrieb freigegebenen Releases veröffentlicht und in Betrieb genommen werden.


\subsection{Nutzung von Statischer Code Analyse}
Die Nutzung von statischer Code-Analyse ist ein essenzieller Bestandteil moderner Softwareentwicklungsprozesse, um die Codequalität und Sicherheit zu gewährleisten. Es gibt eine Vielzahl von Tools zur Durchführung dieser Analysen, darunter SonarQube, ESLint, Pylint und Coverity. Die Autoren dieses Kapitels fokussieren sich auf die Verwendung von SonarQube, basierend auf ihrer umfangreichen Erfahrung und den Vorteilen, die SonarQube bietet. SonarQube wird häufig gewählt, weil es eine umfassende Analyse für eine Vielzahl von Programmiersprachen bietet und sich nahtlos in verschiedene CI/CD-Tools integrieren lässt.

Durch die Integration von SonarQube in den CI/CD-Prozess kann diese Analyse automatisiert und kontinuierlich durchgeführt werden. Innerhalb des GitFlow-Workflows ermöglicht es die Einrichtung von Quality Gates in SonarQube, die spezifische Richtlinien und Kriterien definieren, die der Code erfüllen muss. Wenn diese Quality Gates durch die in SonarQube hinterlegten Policies fehlschlagen, wird der Build-Prozess unterbrochen und ein Merge in ein Release verhindert. Dadurch wird sichergestellt, dass nur qualitativ hochwertiger und sicherer Code in die Produktionsumgebung gelangt, wodurch das Risiko von Sicherheitslücken und anderen Problemen erheblich reduziert wird.

Die Integration von SonarQube in GitHub Actions ermöglicht eine nahtlose Automatisierung dieser Prozesse. Ein typisches Setup könnte wie folgt aussehen:

\begin{itemize}
\item SonarQube Scanner Action: Eine GitHub Action, die SonarQube-Scans ausführt, wird im Workflow definiert. Diese Action wird typischerweise nach dem Build-Schritt und vor dem Deployment-Schritt ausgeführt.
\item Quality Gates: SonarQube überprüft den Code gegen die definierten Quality Gates. Wenn der Code die Kriterien nicht erfüllt, schlägt die Action fehl.
\item Berechtigungen anpassen: Für den Release-Branch müssen möglicherweise die Berechtigungen angepasst werden, um sicherzustellen, dass nur Builds, die die Quality Gates bestehen, gemergt werden können. Dies kann durch die Einrichtung von Branch Protection Rules in GitHub erreicht werden, die verhindern, dass fehlerhafter Code in den Release-Branch gelangt.
\end{itemize}

Durch diese Integration und die Anpassung der Berechtigungen für den Release-Branch wird sichergestellt, dass nur geprüfter und qualitativ hochwertiger Code in die Produktionsumgebung gelangt, was die Sicherheit und Zuverlässigkeit der Software erheblich verbessert.


\subsection{Dependency Scanning und Überprüfung auf Sicherheitslücken}
Dependency Scanning und die Überprüfung auf Sicherheitslücken sind kritische Schritte im Softwareentwicklungsprozess, um sicherzustellen, dass alle verwendeten Bibliotheken und Abhängigkeiten sicher und auf dem neuesten Stand sind. Es gibt mehrere Tools, die für diesen Zweck verwendet werden können, darunter Sonatype Nexus IQ, Snyk und OWASP Dependency-Check. Die Autoren dieses Kapitels fokussieren sich auf die Verwendung von Sonatype Nexus IQ aufgrund ihrer umfassenden Erfahrung und der weitreichenden Fähigkeiten dieses Tools.

Sonatype Nexus IQ bietet umfassende Unterstützung für eine Vielzahl von Programmiersprachen und Paketmanagern, darunter Java (Maven, Gradle), JavaScript (npm, Yarn), Python (pip), Ruby (RubyGems), und viele mehr. Dieses Tool ermöglicht es Entwicklern, Sicherheitslücken in ihren Abhängigkeiten frühzeitig zu erkennen und zu beheben, bevor diese in die Produktion gelangen.

Die Integration von Sonatype Nexus IQ in den CI/CD-Prozess kann automatisiert werden, um kontinuierlich die verwendeten Abhängigkeiten zu scannen und zu überprüfen. Dies kann zum Beispiel durch die Nutzung von GitHub Actions realisiert werden. Ein typischer Workflow könnte wie folgt aussehen:

\begin{itemize}
\item Nexus IQ Scan Action: Eine GitHub Action, die einen Scan mit Nexus IQ ausführt, wird im Workflow definiert. Diese Action wird typischerweise nach dem Build-Schritt und vor dem Deployment-Schritt ausgeführt.
\item Policy Enforcement: Sonatype Nexus IQ überprüft die Abhängigkeiten gegen die definierten Sicherheitsrichtlinien. Wenn eine Sicherheitslücke entdeckt wird, schlägt die Action fehl und verhindert so einen Merge oder Release des fehlerhaften Codes.
\item Berechtigungen anpassen: Für den Release-Branch müssen möglicherweise die Berechtigungen angepasst werden, um sicherzustellen, dass nur Builds, die die Sicherheitsrichtlinien bestehen, gemergt werden können. Dies kann durch die Einrichtung von Branch Protection Rules in GitHub erreicht werden, die verhindern, dass fehlerhafte Abhängigkeiten in den Release-Branch gelangen.
\end{itemize}
\section{Integration von Dependency Scanning und Überprüfung auf Sicherheitslücken in den DevOps-Zyklus}
In diesem Kapitel wird anhand von Dependency Scanning und der Überprüfung auf Sicherheitslücken aufgezeigt, wie Sicherheitslücken in Abhängigkeiten erkannt und behoben werden können, bevor diese in für den Produktionsbetrieb freigegebenen Releases veröffentlicht und in Betrieb genommen werden.

\subsection{Nutzung von Dependency Scanning}
Dependency Scanning und die Überprüfung auf Sicherheitslücken sind kritische Schritte im Softwareentwicklungsprozess, um sicherzustellen, dass alle verwendeten Bibliotheken und Abhängigkeiten sicher und auf dem neuesten Stand sind. Es gibt mehrere Tools, die für diesen Zweck verwendet werden können, darunter Sonatype Nexus IQ, Snyk und OWASP Dependency-Check. Die Autoren dieses Kapitels fokussieren sich auf die Verwendung von Sonatype Nexus IQ aufgrund ihrer umfassenden Erfahrung und der weitreichenden Fähigkeiten dieses Tools.

\subsubsection{Übersicht ähnlicher Produkte}

Es gibt verschiedene Tools, die zur Überprüfung von Abhängigkeiten auf Sicherheitslücken verwendet werden können. Die folgende Tabelle gibt eine Übersicht über einige gängige Tools:

\begin{table*}[h!]
\centering
\begin{tabular}{|l|l|l|l|}
\hline
\textbf{Tool} & \textbf{Programmiersprachen} & \textbf{Lizenz} & \textbf{Besonderheiten} \\ \hline
Sonatype Nexus IQ & Mehrere & Proprietär & Umfassende Sicherheitsanalyse, Policy Enforcement \\ \hline
Snyk & Mehrere & Proprietär & Echtzeit-Schwachstellenüberprüfung, Entwicklungsintegration \\ \hline
OWASP Dependency-Check & Mehrere & Open Source & Fokus auf bekannte Sicherheitslücken \\ \hline
WhiteSource & Mehrere & Proprietär & Automatisierte Open-Source-Sicherheit und Compliance \\ \hline
Black Duck & Mehrere & Proprietär & Umfassende Open-Source-Risikoanalyse \\ \hline
\end{tabular}
\caption{Übersicht von Tools zur Überprüfung von Abhängigkeiten}
\label{tab:dependency_scanning_tools}
\end{table*}

\subsubsection{Funktionsweise von Sonatype Nexus IQ}

Sonatype Nexus IQ ist ein fortschrittliches Tool zur Analyse von Abhängigkeiten und zur Erkennung von Sicherheitslücken. Es arbeitet durch:

\begin{itemize}
    \item \textbf{Scannen der Abhängigkeiten}: Analysiert alle im Projekt verwendeten Abhängigkeiten.
    \item \textbf{Abgleich mit Datenbanken}: Vergleicht Abhängigkeiten mit bekannten Schwachstellen in öffentlichen und privaten Datenbanken.
    \item \textbf{Policy Enforcement}: Überprüft Abhängigkeiten gegen vordefinierte Sicherheitsrichtlinien und Policies.
    \item \textbf{Berichterstellung}: Generiert Berichte und Warnungen über erkannte Sicherheitslücken und Risiken.
\end{itemize}

\begin{figure*}[h!]
\centering
\includegraphics[width=\textwidth]{nexusiq_dashboard.png}
\caption{Beispiel eines Sonatype Nexus IQ Dashboards}
\label{fig:nexusiq_dashboard}
\end{figure*}

\subsubsection{Vorteile von Sonatype Nexus IQ}

Die Verwendung von Sonatype Nexus IQ bietet zahlreiche Vorteile:

\begin{itemize}
    \item \textbf{Umfassende Sicherheitsanalyse}: Erkennt eine Vielzahl von Schwachstellen in Abhängigkeiten.
    \item \textbf{Automatisierte Policy Enforcement}: Stellt sicher, dass nur sichere und konforme Abhängigkeiten verwendet werden.
    \item \textbf{Integration in CI/CD}: Kann nahtlos in bestehende CI/CD-Pipelines integriert werden.
    \item \textbf{Detaillierte Berichte}: Bietet umfangreiche Berichte und Dashboards zur Überwachung der Sicherheit.
    \item \textbf{Regelmäßige Updates}: Erhält kontinuierliche Updates zu neuen Sicherheitslücken und Bedrohungen.
\end{itemize}

\subsubsection{Beispiel Implementierung für GitHub Actions}

Ein typisches Setup für die Integration von Sonatype Nexus IQ in GitHub Actions könnte wie folgt aussehen:

\paragraph{Einrichtung der GitHub Actions Workflow-Datei}

Erstellen Sie im Wurzelverzeichnis Ihres Repositories eine Datei namens \texttt{.github/workflows/nexus-iq.yml} mit folgendem Inhalt:

\begin{lstlisting}
name: Nexus IQ Analysis

on:
  push:
    branches:
      - main
      - develop
  pull_request:
    branches:
      - main
      - develop

jobs:
  nexusIQ:
    runs-on: ubuntu-latest

    steps:
    - name: Check out repository
      uses: actions/checkout@v2

    - name: Set up JDK 11
      uses: actions/setup-java@v1
      with:
        java-version: '11'

    - name: Cache Nexus IQ packages
      uses: actions/cache@v2
      with:
        path: ~/.nexus/cache
        key: ${{ runner.os }}-nexus-cache
        restore-keys: ${{ runner.os }}-nexus-cache

    - name: Install dependencies
      run: ./gradlew build -x test

    - name: Run Nexus IQ analysis
      env:
        NEXUS_IQ_URL: ${{ secrets.NEXUS_IQ_URL }}
        NEXUS_IQ_USERNAME: ${{ secrets.NEXUS_IQ_USERNAME }}
        NEXUS_IQ_PASSWORD: ${{ secrets.NEXUS_IQ_PASSWORD }}
      run: ./gradlew nexusIqScan -Dsonar.projectKey=my_project_key -Dsonar.host.url=${{ secrets.NEXUS_IQ_URL }} -Dsonar.login=${{ secrets.NEXUS_IQ_USERNAME }} -Dsonar.password=${{ secrets.NEXUS_IQ_PASSWORD }}
\end{lstlisting}

\subsubsection{Integration in den GitFlow}

Die Integration von Dependency Scanning in den GitFlow-Prozess folgt einem ähnlichen Muster wie bei der statischen Code Analyse:

\begin{itemize}
    \item \textbf{Feature-Branches}: Scannen der Abhängigkeiten während der Entwicklung neuer Features.
    \item \textbf{Develop-Branch}: Regelmäßiges Scannen vor dem Mergen von Feature-Branches.
    \item \textbf{Release-Branches}: Finales Scannen vor der Freigabe neuer Releases.
    \item \textbf{Hotfix-Branches}: Schnelles Scannen bei dringenden Fehlerbehebungen.
\end{itemize}

\paragraph{Beispielhafte Umsetzung in GitHub}

\begin{enumerate}
    \item Forken des Repository und erstellen eines neuen Branch: \texttt{git checkout -b feature/new-feature develop}
    \item Implementierung des neuen Features und push der Änderungen: \texttt{git push origin feature/new-feature}
    \item Öffnen eines Pull Request von \texttt{feature/new-feature} nach \texttt{develop}.
    \item Die CI/CD-Pipeline führt das Dependency Scanning mit Sonatype Nexus IQ durch.
    \item Bestehen alle Checks, kann der Branch gemerged werden.
\end{enumerate}

\subsubsection{Sicherheitsgewinn durch Dependency Scanning}

Die Integration von Dependency Scanning und der Überprüfung auf Sicherheitslücken in den CI/CD-Prozess bietet erhebliche Sicherheitsvorteile:

\begin{itemize}
    \item \textbf{Früherkennung von Sicherheitslücken}: Sicherheitslücken in Abhängigkeiten werden frühzeitig erkannt und können behoben werden, bevor sie in die Produktionsumgebung gelangen.
    \item \textbf{Kontinuierliche Überwachung}: Durch kontinuierliches Scannen werden neue Schwachstellen sofort erkannt.
    \item \textbf{Compliance und Sicherheit}: Durchsetzen von Sicherheitsrichtlinien und Compliance-Anforderungen.
\end{itemize}

\subsubsection{Warnhinweise und Einschränkungen}

Es ist wichtig zu beachten, dass trotz der Verwendung von Tools wie Sonatype Nexus IQ Sicherheitslücken in Abhängigkeiten dennoch vorhanden sein können. Entwickler sollten sich dessen bewusst sein und folgende Vorsichtsmaßnahmen treffen:

\begin{itemize}
    \item \textbf{Regelmäßige Updates}: Stellen Sie sicher, dass alle Abhängigkeiten regelmäßig auf die neuesten Versionen aktualisiert werden.
    \item \textbf{Manuelle Überprüfungen}: Ergänzen Sie automatisierte Scans durch manuelle Überprüfungen und Sicherheitsaudits.
    \item \textbf{Bewusstsein und Schulung}: Schulen Sie das Entwicklungsteam in Best Practices der sicheren Softwareentwicklung.
    \item \textbf{Kontinuierliche Verbesserung}: Passen Sie Sicherheitsrichtlinien und -prozesse kontinuierlich an neue Bedrohungen und Schwachstellen an.
\end{itemize}

\subsubsection{Details zu OWASP}

Die Open Web Application Security Project (OWASP) bietet eine Vielzahl von Ressourcen zur Verbesserung der Software-Sicherheit. Das OWASP Top Ten Projekt ist eine jährlich aktualisierte Liste der am häufigsten auftretenden und kritischsten Sicherheitsrisiken für Webanwendungen. Einige der in Sonatype Nexus IQ integrierten OWASP-Regeln umfassen:

\begin{itemize}
    \item \textbf{Injection}: Schutz vor verschiedenen Injektionstypen, einschließlich SQL-Injection.
    \item \textbf{Broken Authentication}: Überprüfung auf Schwachstellen in der Authentifizierung.
    \item \textbf{Sensitive Data Exposure}: Sicherstellung der Verschlüsselung sensibler Daten.
    \item \textbf{XML External Entities (XXE)}: Erkennung von Schwachstellen, die durch externe XML-Entitäten verursacht werden.
    \item \textbf{Broken Access Control}: Verhinderung von unbefugtem Zugriff auf Ressourcen.
\end{itemize}

\subsubsection{Schlussfolgerungen und Best Practices}

Die Integration von Dependency Scanning in den CI/CD-Prozess verbessert nicht nur die Sicherheit der Software, sondern trägt auch zur Einhaltung von Compliance-Richtlinien bei. Tools wie Sonatype Nexus IQ bieten umfassende Analysen und automatisierte Policy Enforcement, um Sicherheitslücken frühzeitig zu erkennen und zu beheben. Es ist jedoch entscheidend, Sicherheitsmaßnahmen kontinuierlich zu überwachen und anzupassen, um den höchsten Sicherheitsstandard zu gewährleisten.
\section{Fazit}
Die Implementierung von statischer Code-Analyse und Dependency Scanning in den CI/CD-Prozess ist von entscheidender Bedeutung für die Sicherstellung der Codequalität und -sicherheit. Durch den Einsatz von SonarQube und Sonatype Nexus IQ können Entwickler frühzeitig potenzielle Probleme und Sicherheitslücken identifizieren und beheben. Die kontinuierliche Überprüfung und Durchsetzung von Qualitäts- und Sicherheitsrichtlinien tragen dazu bei, dass nur geprüfter und sicherer Code in die Produktionsumgebung gelangt. Die Anpassung der Berechtigungen für Release-Branches und die Einrichtung von automatisierten Workflows in GitHub Actions sind wichtige Schritte, um einen reibungslosen und sicheren Entwicklungsprozess zu gewährleisten. Letztendlich führen diese Maßnahmen zu einer höheren Softwarequalität, die den Anforderungen moderner Anwendungen gerecht wird.




% -------------------------------------------------------







% --------------------------------------------------------------------
\section*{Abkürzungen}
\addcontentsline{toc}{section}{Abkürzungen}

% Die längste Abkürzung wird in die eckigen Klammern
% bei \begin{acronym} geschrieben, um einen hässlichen
% Umbruch zu verhindern
% Sie müssen die Abkürzungen selbst alphabetisch sortieren!
\begin{acronym}[IEEE]
\acro{A2A}{Application-to-Application}
\acro{ABK}{Abkürzung}
\acro{ACL}{Acess Control List}
\acro{ACM}{Association of Computing Machinery}
\acro{AES}{Advanced Encryption Standard}
\acro{IEEE}{Institute of Electrical and Electronics Engineers}
\acro{ISO}{International Organization for Standardization}
\acro{PDF}{Portable Document Format}
\end{acronym}

% Literaturverzeichnis
\addcontentsline{toc}{section}{Literatur}
\printbibliography

\end{document}
