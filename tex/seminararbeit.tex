\documentclass[conference,compsoc,final,a4paper]{IEEEtran}
\usepackage[utf8]{inputenx}

%% Bitte legen Sie hier den Titel und den Autor der Arbeit fest
\newcommand{\autoren}[0]{Kari, Jonas \newline Rittershofer, Daniel}

\newcommand{\dokumententitel}[0]{DevSecOps - Implementierung von Quality-Gates zur Förderung der Sicherheit von Software-Artefakten während der Entwicklung}

\input{preambel} % Weitere Einstellungen aus einer anderen Datei lesen

\begin{document}

% Titel des Dokuments
\title{\dokumententitel}

% Namen der Autoren
\author{
  \IEEEauthorblockN{\autoren}
  \IEEEauthorblockA{
    Hochschule Mannheim\\
    Fakultät für Informatik\\
    Paul-Wittsack-Str. 10,
    68163 Mannheim
    }
}

% Titel erzeugen
\maketitle
\thispagestyle{plain}
\pagestyle{plain}

% Eigentliches Dokument beginnt hier
% ----------------------------------------------------------------------------------------------------------

% Kurze Zusammenfassung des Dokuments
\begin{abstract}
TBD

\end{abstract}

% Inhaltsverzeichnis erzeugen
{\small\tableofcontents}

% Abschnitte mit \section, Unterabschnitte mit \subsection und
% Unterunterabschnitte mit \subsubsection
% -------------------------------------------------------
\section{Entstehung von DevOps und der Weg zu DevSecOps}
In der heutigen Zeit gewinnt das DevOps-Prinzip im Softwareentwicklungszyklus zunehmend an Bedeutung. DevOps fördert die Integration von Entwicklung und Betrieb und verkürzt die bisherige Trennung zwischen diesen Bereichen. Dieser Ansatz wird durch das Prinzip "you build it, you run it" verkörpert, welches die Verantwortlichkeit für den gesamten Lebenszyklus einer Anwendung in die Hände der Entwickler legt.

Ein wesentlicher Aspekt dieser Integration ist die Sicherheit, die traditionell als separater Schritt behandelt wurde. Der "shift left" Ansatz integriert Sicherheitsaspekte frühzeitig und kontinuierlich in den Softwareentwicklungsprozess. Dies bedeutet, dass Sicherheitsmaßnahmen nicht nur in der Entwicklungsphase, sondern auch während des Betriebs berücksichtigt werden müssen.

In dieser Arbeit liegt der Schwerpunkt auf der Implementierung von Quality Gates, die darauf abzielen, die Sicherheit von Software zu erhöhen. Insbesondere wird der Fokus auf das Dependency Scanning zur Identifikation bekannter Sicherheitslücken und die statische Code-Analyse gelegt, um potenzielle Fehler und Schwachstellen im Code frühzeitig zu erkennen.

Durch diese Maßnahmen wird nicht nur die Sicherheit der Software verbessert, sondern auch die Effizienz des gesamten Entwicklungsprozesses gesteigert. Die Bedeutung dieser Ansätze wird im Kontext moderner Softwareentwicklung erläutert und durch praxisnahe Beispiele und Fallstudien untermauert.


Diese Arbeit richtet sich an Software-Entwickler und IT-Fachleute, die DevOps und DevSecOps einführen möchten. Ziel ist es, die Sicherheit und Qualität ihrer Software-Releases zu verbessern. Leser sollten grundlegendes Wissen über Softwareentwicklung und IT-Betrieb haben und daran interessiert sein, Sicherheits- und Qualitätsstrategien sowie Security-Gates in ihren Entwicklungsprozess zu integrieren. Die Arbeit bietet anhand von zwei Beispielen eine Implementierung sowie Best Practices, um eine sichere und effiziente Softwareentwicklung zu gewährleisten.


% -------------------------------------------------------
\section{Historische Entwicklung von DevOps zu DevSecOps}
Dieses Kapitel zeigt die Entstehung der DevOps-Kultur auf und wie der logische Schritt zu DevSecOps vollzogen wurde.

\subsection{Entstehung von DevOps und der Weg zu DevSecOps}
Der Begriff DevOps entstand Ende der 2000er Jahre und beschreibt die engere Zusammenarbeit von Software-Entwicklung und IT-Betrieb. In der Vergangenheit – und teilweise auch heute noch – führten die Kluft zwischen Abteilungen und unklare Zuständigkeiten oft zu Problemen, die erst dann sichtbar wurden, wenn eine Abteilung, die nicht in die Entwicklung involviert war, eine Software oder ein Softwaresystem mit meist manuellen Schritten in den Produktivbetrieb übernehmen sollte. Durch das DevOps-Prinzip können solche Probleme frühzeitig erkannt werden, insbesondere durch ein T-Shaped-Entwicklungsteam, in dem auch Kenntnisse im Betrieb von Software vorhanden sind.

DevOps ist als eine Kultur und Praxis zu verstehen, die Entwicklung und Betrieb integriert. Ziel ist es, durch automatisierte Prozesse und verbesserte Zusammenarbeit eine qualitativ hochwertige und reproduzierbar betreibbare Software schneller bereitzustellen.

\subsection{TBD: Zusammenarbeit zwischen Entwicklung und Betrieb}
Traditionell arbeiteten Entwicklung und Betrieb isoliert, was zu Kommunikationsproblemen und Verzögerungen führte. DevOps fördert die enge Zusammenarbeit beider Teams, um schnell auf Marktveränderungen zu reagieren und die Softwarequalität zu verbessern.


\subsection{TBD: Prinzipien der kontinuierlichen Integration und kontinuierlichen Bereitstellung (CI/CD)}
Kontinuierliche Integration (CI): Entwickler integrieren regelmäßig ihren Code in ein zentrales Repository, gefolgt von automatisierten Builds und Tests, um Fehler frühzeitig zu erkennen und zu beheben.

Kontinuierliche Bereitstellung (CD): Automatisiert die Bereitstellung von Softwareänderungen in verschiedene Umgebungen, sodass jede erfolgreiche Änderung schnell und zuverlässig an Endnutzer ausgeliefert werden kann.




% -------------------------------------------------------
\section{Integration von Sicherheitspraktiken in den DevOps-Zyklus}
In diesem Kapitel wird anhand von statischer Code-Analyse und Dependency Scanning aufgezeigt, wie Sicherheitslücken erkannt und behoben werden können, bevor diese in für den Produktionsbetrieb freigegebenen Releases veröffentlicht und in Betrieb genommen werden.


\subsection{Nutzung von Statischer Code Analyse}
Die Nutzung von statischer Code-Analyse ist ein essenzieller Bestandteil moderner Softwareentwicklungsprozesse, um die Codequalität und Sicherheit zu gewährleisten. Es gibt eine Vielzahl von Tools zur Durchführung dieser Analysen, darunter SonarQube, ESLint, Pylint und Coverity. Die Autoren dieses Kapitels fokussieren sich auf die Verwendung von SonarQube, basierend auf ihrer umfangreichen Erfahrung und den Vorteilen, die SonarQube bietet. SonarQube wird häufig gewählt, weil es eine umfassende Analyse für eine Vielzahl von Programmiersprachen bietet und sich nahtlos in verschiedene CI/CD-Tools integrieren lässt.

Durch die Integration von SonarQube in den CI/CD-Prozess kann diese Analyse automatisiert und kontinuierlich durchgeführt werden. Innerhalb des GitFlow-Workflows ermöglicht es die Einrichtung von Quality Gates in SonarQube, die spezifische Richtlinien und Kriterien definieren, die der Code erfüllen muss. Wenn diese Quality Gates durch die in SonarQube hinterlegten Policies fehlschlagen, wird der Build-Prozess unterbrochen und ein Merge in ein Release verhindert. Dadurch wird sichergestellt, dass nur qualitativ hochwertiger und sicherer Code in die Produktionsumgebung gelangt, wodurch das Risiko von Sicherheitslücken und anderen Problemen erheblich reduziert wird.

Die Integration von SonarQube in GitHub Actions ermöglicht eine nahtlose Automatisierung dieser Prozesse. Ein typisches Setup könnte wie folgt aussehen:

\begin{itemize}
\item SonarQube Scanner Action: Eine GitHub Action, die SonarQube-Scans ausführt, wird im Workflow definiert. Diese Action wird typischerweise nach dem Build-Schritt und vor dem Deployment-Schritt ausgeführt.
\item Quality Gates: SonarQube überprüft den Code gegen die definierten Quality Gates. Wenn der Code die Kriterien nicht erfüllt, schlägt die Action fehl.
\item Berechtigungen anpassen: Für den Release-Branch müssen möglicherweise die Berechtigungen angepasst werden, um sicherzustellen, dass nur Builds, die die Quality Gates bestehen, gemergt werden können. Dies kann durch die Einrichtung von Branch Protection Rules in GitHub erreicht werden, die verhindern, dass fehlerhafter Code in den Release-Branch gelangt.
\end{itemize}

Durch diese Integration und die Anpassung der Berechtigungen für den Release-Branch wird sichergestellt, dass nur geprüfter und qualitativ hochwertiger Code in die Produktionsumgebung gelangt, was die Sicherheit und Zuverlässigkeit der Software erheblich verbessert.


\subsection{Dependency Scanning und Überprüfung auf Sicherheitslücken}
Dependency Scanning und die Überprüfung auf Sicherheitslücken sind kritische Schritte im Softwareentwicklungsprozess, um sicherzustellen, dass alle verwendeten Bibliotheken und Abhängigkeiten sicher und auf dem neuesten Stand sind. Es gibt mehrere Tools, die für diesen Zweck verwendet werden können, darunter Sonatype Nexus IQ, Snyk und OWASP Dependency-Check. Die Autoren dieses Kapitels fokussieren sich auf die Verwendung von Sonatype Nexus IQ aufgrund ihrer umfassenden Erfahrung und der weitreichenden Fähigkeiten dieses Tools.

Sonatype Nexus IQ bietet umfassende Unterstützung für eine Vielzahl von Programmiersprachen und Paketmanagern, darunter Java (Maven, Gradle), JavaScript (npm, Yarn), Python (pip), Ruby (RubyGems), und viele mehr. Dieses Tool ermöglicht es Entwicklern, Sicherheitslücken in ihren Abhängigkeiten frühzeitig zu erkennen und zu beheben, bevor diese in die Produktion gelangen.

Die Integration von Sonatype Nexus IQ in den CI/CD-Prozess kann automatisiert werden, um kontinuierlich die verwendeten Abhängigkeiten zu scannen und zu überprüfen. Dies kann zum Beispiel durch die Nutzung von GitHub Actions realisiert werden. Ein typischer Workflow könnte wie folgt aussehen:

\begin{itemize}
\item Nexus IQ Scan Action: Eine GitHub Action, die einen Scan mit Nexus IQ ausführt, wird im Workflow definiert. Diese Action wird typischerweise nach dem Build-Schritt und vor dem Deployment-Schritt ausgeführt.
\item Policy Enforcement: Sonatype Nexus IQ überprüft die Abhängigkeiten gegen die definierten Sicherheitsrichtlinien. Wenn eine Sicherheitslücke entdeckt wird, schlägt die Action fehl und verhindert so einen Merge oder Release des fehlerhaften Codes.
\item Berechtigungen anpassen: Für den Release-Branch müssen möglicherweise die Berechtigungen angepasst werden, um sicherzustellen, dass nur Builds, die die Sicherheitsrichtlinien bestehen, gemergt werden können. Dies kann durch die Einrichtung von Branch Protection Rules in GitHub erreicht werden, die verhindern, dass fehlerhafte Abhängigkeiten in den Release-Branch gelangen.
\end{itemize}


\section{Fazit}
Die Implementierung von statischer Code-Analyse und Dependency Scanning in den CI/CD-Prozess ist von entscheidender Bedeutung für die Sicherstellung der Codequalität und -sicherheit. Durch den Einsatz von SonarQube und Sonatype Nexus IQ können Entwickler frühzeitig potenzielle Probleme und Sicherheitslücken identifizieren und beheben. Die kontinuierliche Überprüfung und Durchsetzung von Qualitäts- und Sicherheitsrichtlinien tragen dazu bei, dass nur geprüfter und sicherer Code in die Produktionsumgebung gelangt. Die Anpassung der Berechtigungen für Release-Branches und die Einrichtung von automatisierten Workflows in GitHub Actions sind wichtige Schritte, um einen reibungslosen und sicheren Entwicklungsprozess zu gewährleisten. Letztendlich führen diese Maßnahmen zu einer höheren Softwarequalität, die den Anforderungen moderner Anwendungen gerecht wird.



% --------------------------------------------------------------------
\section*{Abkürzungen}
\addcontentsline{toc}{section}{Abkürzungen}

% Die längste Abkürzung wird in die eckigen Klammern
% bei \begin{acronym} geschrieben, um einen hässlichen
% Umbruch zu verhindern
% Sie müssen die Abkürzungen selbst alphabetisch sortieren!
\begin{acronym}[IEEE]
\acro{A2A}{Application-to-Application}
\acro{ABK}{Abkürzung}
\acro{ACL}{Acess Control List}
\acro{ACM}{Association of Computing Machinery}
\acro{AES}{Advanced Encryption Standard}
\acro{IEEE}{Institute of Electrical and Electronics Engineers}
\acro{ISO}{International Organization for Standardization}
\acro{PDF}{Portable Document Format}
\end{acronym}

% Literaturverzeichnis
\addcontentsline{toc}{section}{Literatur}
\printbibliography

\end{document}
